\documentclass{beamer}

\usepackage[utf8]{inputenc}
\usepackage[T1]{fontenc}
\usepackage{xcolor}
\usepackage{changepage}
\usepackage{textcomp}
\usepackage{hyperref}
\usepackage[frenchb]{babel}

\usetheme{Hannover}

\beamertemplatenavigationsymbolsempty
\setbeamerfont{page number in foot}{size=\large}
\setbeamertemplate{footline}[frame number]

\title{Introduction à Docker}
\subtitle{DevLab Apside}
\date{7 février 2019}
\titlegraphic{\includegraphics[height=4cm]{img/docker.png}}

\begin{document}

\begin{frame}
  \titlepage
\end{frame}

\begin{frame}{Sommaire}
\tableofcontents
\end{frame}

\begin{section}{Docker, c'est quoi ?}
\begin{frame}{Un peu d'histoire}
    \begin{itemize}
        \item mars 2013 : Release par Solomon Hykes (dotcloud)
        \item octobre 2013 : dotcloud devient Docker, Inc
        \item 2014 : passage de Linux containers à libcontainers (Golang)
        \item 2014 : partenariat avec Amazon EC2 et IBM
        \item 2015 : Succès Github (plus de 1100 contributeurs)
        \item 2016 : Windocks, portage du projet à Windows 
        \item 2017 : 13 milliards de téléchargements, + 160\% de mentions sur LinkedIn par rapport à 2016
        \item 2019 : 1917 contributeurs, 43 573 commits
    \end{itemize}
\end{frame}

\begin{frame}{En bref}
    \begin{itemize}
        \item Projet Open Source
        \item Gestionnaire de containers
        \item Peut tourner dans une machine virtuelle
        \item Permet de faire tourner des micro services sur une architecture distribuée
    \end{itemize}
\end{frame}

\begin{frame}{Container et machine virtuelle}
    \begin{block}{Container}
         Un container est un \textbf{package exécutable, léger et autonome} qui contient tout ce qu'il faut pour faire tourner un logiciel et qui est \textbf{indépendant du système d'exploitation}.
    \end{block}
    
    \begin{block}{Machine Virtuelle}
         Une machine virtuelle est une \textbf{émulation de ressources matérielles et logicielles} telles que la mémoire, le processeur, le disque dur, et le système d'exploitation qui permet d'exécuter les programmes dans les mêmes conditions que la machine simulée. Ça permet une grande \textbf{portabilité des logiciels}.
    \end{block}
\end{frame}

\begin{frame}{Containers vs Machine Virtuelle}
    \includegraphics[width=10cm]{img/dockerVSVM.jpg}
\end{frame}

\begin{frame}{Architecture}
    \includegraphics[width=10cm]{img/architecture.png}
\end{frame}

\begin{frame}{Organisation courante}
    \includegraphics[width=10cm]{img/docker_process.png}
\end{frame}
\end{section}

\begin{section}{DockerHub}
\begin{frame}{DockerHub}
    \begin{itemize}
        \item Registre d'images publiques docker
        \item \href{https://hub.docker.com}{https://hub.docker.com}
        \item Cible par défaut de docker pull
        \item Permet également de partager des images
    \end{itemize}
\end{frame}
\end{section}

\begin{section}{Mise en place et Utilisation}

\end{section}
\begin{section}{Limites}

\end{section}
\begin{section}{Problèmes courants}

\end{section}
\begin{section}{Exemples de Dockerfile}

\end{section}

\begin{section}{Pour aller plus loin}



\end{section}

\begin{frame}{Merci !}
    Merci de votre attention, si vous avez des questions, n'hésitez pas.
    
    Ces slides sont disponibles sur \href{https://github.com/freyj/}{https://github.com/freyj/docker-intro} 
    avec un exemple de DockerFile ainsi que les sources qui ont servi à cette présentation.
    
    \begin{columns}
        \begin{column}{0.5\textwidth}
            \includegraphics[height=3cm]{img/biere.jpg}
        \end{column}
        \begin{column}{0.5\textwidth}
            \includegraphics[height=3cm]{img/pizza.jpg}
        \end{column}
    \end{columns}
\end{frame}

 \begin{frame}{Sources}
 \begin{itemize}
     \item \href{https://en.wikipedia.org/wiki/Docker_(software)}{Page wikipedia sur Docker}
     \item \href{https://www.docker.com/}{Site de Docker, Inc}
     \item \href{https://github.com/docker/docker-ce}{Repo github de docker}
     \item \href{https://www.aquasec.com/wiki/display/containers/}{Documentation sur les containers}
     \item \href{https://docs.docker.com/develop}{La documentation docker officielle}
 \end{itemize}
 \end{frame}




\end{document}


